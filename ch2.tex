\chapter{Computational Neuroscience}
\label{chap:computation_neuroscience}
Computational neuroscience is an interdisciplinary field that
utilizes mathematics and computer science knowledge to better
understand the nervous system and brain. With the rampant
ascent of the computational efficiency and technology for
obtaining more data from the experiments this field
became more and more relevant. Especially, in the last
years with the ascent of the DNNs. The models are great due for
its replicability, stability. It does not contain any noise and so on.

We divide the field to several different approaches of 
how we model the system. We can model it exploiting the numerical
results from the experiments to optimize the system to correspond 
these data. On the other hand, we can also try to aim to design
the model that tries to mimic the biological behavior as much as possible.

In this thesis, we try the second approach. We exploit the 
RNN models and known biological constraints to develop reasonably explainable
network where each of the neurons correspond to the real life neuron.