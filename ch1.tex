\chapter{Early Visual System}
\label{chap:visual_system}
In our thesis we are focussing in modeling the primary visual cortex (V1)
which plays a significant role in the early visual processing in the
mammalian brain. This area is widely studied and has been a subject of
research for many decades. In this chapter we will focus mainly on the 
parts closely related to our work. Thus, the comprehensive description
of the system can be seen for example in the "Bear book". 

\section{General Structure of the Early Visual System}
\label{sec:general_structure}
Overall, the areas responsible for the visual processing covers nearly 
a half of the human brain. The first stage of the visual processing happens in
the so called Early Visual System.

The visual signal travels in form of light through the complex light modulating
system of the eye and is projected onto the retina. The retina is a thin layer
of neural tissue that is responsible for light detection and initial processing.
The signal then travels through the optic nerve to the optic chiasm, where the 
signals are split based on the visual field side, and then travel through the
optic tract to the lateral geniculate nucleus (LGN) of the thalamus. The LGN is
responsible fort the initial processing of the visual signal and then sends majority
of the signal to the primary visual cortex (V1) through the optic radiation. The
V1 is the first cortical area that is fully specialized for visual processing. 
First more complex computations are done there. When the signal leaves the V1,
it is further processed int the higher visual areas that are responsible for the
high level visual processing.

\section{Eye}
\label{sec:eye}
One of the key objectives of an eye is to collect the electromagnetic signal
from the environment bundle it using the complex optical system and project it
onto the retina. The eye comprises of several specialized parts that appropriately
modulates and bends the light. For the comprehensive descriptions of these 
parts see image (TODO: image with eye description) or the appropriate literature
(TODO: citation to comprehensive eye description).

\subsection{Retina}
\label{subsec:retina}
Second important function of the eye is to convert and preprocess the 
light signal into the neural signal that can be further processed in the brain.
This is done by the retina. It is thin layer of neural tissue located at the back
of eye. It is composed of several types of cells and layers each of them responsible
for different part of signal modulation. These cells are visual receptors,
bipolar cells, horizontal cells, amacrine cells and ganglion cells.

The first cells in the manaegerie of visual processing are the \emph{photoreceptor cells}
called \emph{rods} and \emph{cones}. These cells exploits the photochemical reaction of the 
proteins located in their outer segments to convert the light signal into the
chemical signal. The rods are very sensitive to small light intensities, 
detect wide range of wavelengths and are responsible for the low light vision.
On the other hand cone are selective only for the specific wavelengths and their
sensitivity to lower light intensities is much lower. They are responsible for the
color and acuity vision. The distribution of these cells is not uniform across
the retina. The fovea, the central part of the retina, is rich in cones. On the other
hand the periphery of the retina is mainly populated by the rods. It is worth
mentioning that there there is also a small region called the blind spot where
the optic nerve leaves the eye and there are thus no photoreceptors. The 
visual description of the photoreceptor distribution and retina structure we
provide the image (TODO: image of the retina, photoreceptor distribution). Finally,
it is worth noting tha that when the photoreceptors are activated, they hyperpolarized
which results in releasing the neurotransmitter glutamate. This neurotransmitter
is then detected by the bipolar cells. This feature in some modifications is 
typical for all the retina cells except the ganglion cells that elicits the 
action potential typical for the neural cells.

The signal from the photoreceptors is then further processed by the 
\emph{bipolar cells}. Their amount is around 10M and they summarize the 
information from around 125 millions of photoreceptors. The amount of the 
photoreceptors they connect is dependent on the location and type of the 
photoreceptor it connects. These cells closely cooperate with the 
\emph{horizontal cells} that overarches wider range of the photoreceptors
and provides the signal about surrounding light intensity to bipolar cells.

The bipolar cells divided into two types: \emph{ON bipolar cells} and 
\emph{OFF bipolar cells}. The difference between these two types is in the 
way they react to the light signal. The ON bipolar cells are activated when there
is a light signal present, while the OFF bipolar cells are activated when there is not
the signal. There is the 
