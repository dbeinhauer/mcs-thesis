\chapter{Early Visual System}
\label{chap:visual_system}
In our thesis we are focussing in modeling the primary visual cortex (V1)
which plays a significant role in the early visual processing in the
mammalian brain. This area is widely studied and has been a subject of
research for many decades. In this chapter we will focus mainly on the 
parts closely related to our work. Thus, the comprehensive description
of the system can be seen for example in the "Bear book". 

\section{Neuron}
\label{sec:neuron}
The elementary part of the neural processing is a specialized cell called \emph{neuron}.
It consists of three parts the soma, the axon and the dendrites. The soma is a
part structurally similar to other cells. The dendrites are specialized structures
for signal detection from other cells. Finally, the axon is highly specialized
elongated structure that serves to pass the information over long distances 
to enable communication between neurons. It starts in the \emph{axon hillock}
from the soma and ends in the \emph{axon terminal} in \emph{synapse}. For its 
unique functionality it consist of wide range of specialized proteins used for 
potential propagation. At the axon terminal there is a high amount of 
\emph{synaptic vesicles} used containing \emph{neurotransmitters} chemicals
used for communication between the neurons in the synapse such as 
glutamate, gamma-aminobutyric acid (GABA), glycine and acetylcholine.

\subsection{Action Potential}
\label{subsec:action_potential}
The main function of neurons is to spread the \emph{action potential} that enables to 
communicate with other neurons or cells. The basic principle stands on the 
differences in ion concentration gradients and electric potential. The key 
terms are \emph{membrane potential} the voltage between inner and outer space 
of the neuron and \emph{equilibrium potentials} the state when ionic concentration
gradient and electrical potential effects eliminate each other. It means that the
ions do not travel to any site. Regarding this it is worth mentioning that the whole
managerie stands on the different concentrations of the Na+, K+, Cl- and Ca2+. All of them
except the potassium are in higher concentration outside the cell. Majority of the 
high brain energy demand is spent on the maintaining of these concentration gradients.

Thanks to this property the resting membrane potential of the neurons is around $-65 mV$.
The whole workflow of the action potential spread is the following. The input signals
from the other neurons appropriately open the ion channels which results in flux
of the ions and change in the electric potential called \emph{generator potential}.
In case it reaches given threshold the \emph{action potential} is elicited. In this
situation there are enough of the sodium ion channels opened to favor the sodium
permeability over potassium. It leads to \emph{rising phase} in which effect of both 
the negative membrane potential and sodium concentration gradient lead to rapid
depolarization of the membrane and also reaches the positive potential values.
This stage is called the \emph{overshoot}. This happens till the voltage
gated potassium channels finally open and consequently sodium channels became
inactivated. The membrane potential starts to hyperpolarize again. This stage is
called the \emph{falling phase}. The membrane potential drops even under the 
resting potential. This phase is called \emph{undershoot}. Then, potassium 
channels became inactivated and the so called \emph{absolute refractory period} 
starts. In this period the potassium channels are still inactivated and 
thus it is not possible to elicit another action potential. While the 
membrane potential regenerates enough it reaches the phase called 
\emph{relative refractory period}. In this phase the membrane is still 
hyperpolarized but it is possible to elicit new action potential. Due
of the depolarization it requires higher level of the generation potential.

It is important to conduct the action potential in one direction typically
from soma to axon terminal. This is done by the high amount of the 
ion channels along the axons that are periodically opened and inactivated
to propagate the signal in the correct way. To speed up the propagation especially
in the long range axons the \emph{myelination} plays crucial role. It serves basically
as the electric insulation with the gaps called the \emph{nodes of Ranvier}. As the 
results the signal can propagate faster.

At the synaptic end of the axon a high amount of the vesicles with the neurotransmitter
are located. They are effluxed by exocytosis in case the action potential reaches the axon.
Based on the action on the postsynaptic neuron we differentiate between \emph{excitatory}
connections that leads to depolarization of the membrane and thus lead towards
action potential and the \emph{inhibitory} connections that suppress the depolarization
and hyperpolarize the target neuron even more.

\section{General Structure of the Early Visual System}
\label{sec:general_structure}
Overall, the areas responsible for the visual processing covers nearly 
a half of the human brain. The first stage of the visual processing happens in
the so called Early Visual System.

The visual signal travels in form of light through the complex light modulating
system of the eye and is projected onto the retina. The retina is a thin layer
of neural tissue that is responsible for light detection and initial processing.
The signal then travels through the optic nerve to the optic chiasm, where the 
signals are split based on the visual field side, and then travel through the
optic tract to the lateral geniculate nucleus (LGN) of the thalamus. The LGN is
responsible fort the initial processing of the visual signal and then sends majority
of the signal to the primary visual cortex (V1) through the optic radiation. The
V1 is the first cortical area that is fully specialized for visual processing. 
First more complex computations are done there. When the signal leaves the V1,
it is further processed int the higher visual areas that are responsible for the
high level visual processing.

\section{Eye}
\label{sec:eye}
One of the key objectives of an eye is to collect the electromagnetic signal
from the environment bundle it using the complex optical system and project it
onto the retina. The eye comprises of several specialized parts that appropriately
modulates and bends the light. For the comprehensive descriptions of these 
parts see image (TODO: image with eye description) or the appropriate literature
(TODO: citation to comprehensive eye description).

\subsection{Retina}
\label{subsec:retina}
Second important function of the eye is to convert and preprocess the 
light signal into the neural signal that can be further processed in the brain.
This is done by the retina. It is thin layer of neural tissue located at the back
of eye. It is composed of several types of cells and layers each of them responsible
for different part of signal modulation. These cells are visual receptors,
bipolar cells, horizontal cells, amacrine cells and ganglion cells.

The first cells in the manaegerie of visual processing are the \emph{photoreceptor cells}
called \emph{rods} and \emph{cones}. These cells exploits the photochemical reaction of the 
proteins located in their outer segments to convert the light signal into the
chemical signal. The rods are very sensitive to small light intensities, 
detect wide range of wavelengths and are responsible for the low light vision.
On the other hand cone are selective only for the specific wavelengths and their
sensitivity to lower light intensities is much lower. They are responsible for the
color and acuity vision. The distribution of these cells is not uniform across
the retina. The fovea, the central part of the retina, is rich in cones. On the other
hand the periphery of the retina is mainly populated by the rods. It is worth
mentioning that there there is also a small region called the blind spot where
the optic nerve leaves the eye and there are thus no photoreceptors. The 
visual description of the photoreceptor distribution and retina structure we
provide the image (TODO: image of the retina, photoreceptor distribution). Finally,
it is worth noting tha that when the photoreceptors are activated, they hyperpolarized
which results in releasing the neurotransmitter glutamate. This neurotransmitter
is then detected by the bipolar cells. This feature in some modifications is 
typical for all the retina cells except the ganglion cells that elicits the 
action potential typical for the neural cells.

The signal from the photoreceptors is then further processed by the 
\emph{bipolar cells}. Their amount is around 10M and they summarize the 
information from around 125 millions of photoreceptors. The amount of the 
photoreceptors they connect is dependent on the location and type of the 
photoreceptor it connects. These cells closely cooperate with the 
\emph{horizontal cells} that overarches wider range of the photoreceptors
and provides the signal about surrounding light intensity to bipolar cells.

The bipolar cells are divided into two types: \emph{ON bipolar cells} and 
\emph{OFF bipolar cells}. The difference between these two types is in the 
way they react to the light signal. The ON bipolar cells are activated when there
is a light signal present, while the OFF bipolar cells are activated when there is not
the signal. They react positively to the light signal from the center of the receptory
field and negatively to the light signal from the surrounding area that is gathered by the
horizontal cells. The receptive field is basically circumscribed area of the photoreceptors. The 
radius changes based on the type and position of in the retina.

The signal then travels to the \emph{ganglion cells} that closely cooperate with the 
amacrine cells. The way principle of the signal processing in this layer is analogical to 
the processing in the bipolar cells and the horizontal cells. The ganglion cells are the 
first cells that elicit the action potential. It is due to the fact that they connect the
retina with the brain and the signal thus need to travel longer distance. The axons of these
cells then terminates in the \emph{Lateral Geniculate Nucleus (LGN)} of the thalamus.

\section{Lateral Geniculate Nucleus}
\label{sec:lgn}
The following steps in the visual processing happen in the 
\emph{Lateral Geniculate Nucleus (LGN)}. It is a part of the thalamus on each side of the 
brain responsible for the initial processing of the visual signal. It is the region where
the majority of the optic tract fibers terminate. Majority of the outputs from this region
then travel to the \emph{primary visual cortex (V1)} through the optic radiation. Where is
the center of the conscious visual perception. The defects in the pathway from the retina 
to the LGN and to the V1 can lead to blindness. Thanks to the structure of the visual
pathway we can tract the defects in the visual system to the specific location. It 
is worth mentioning that majority of the input to the LGN comes from the V1. Currently, 
it is not clear the exact role of these connections but it is believed that they are mainly
responsible for the feedback and modulation of the signal.

The LGN is composed of six layers. These are split into half for each eye. The receptive
fields of the similar to retinal cells concentric.

\section{Primary Visual Cortex}
\label{sec:v1}
The primary visual cortex (V1) is the first cortical area that is fully specialized for 
the visual processing. It is located in the occipital (back) lobe of the brain. The 
V1 is responsible for the initial processing of the visual stimuli. It encodes for example
the orientation, size, motion and color. The output is then split into the two streams
the P-Stream responsible for orientation and location and the M-stream responsible for 
the motion.

The V1 is composed of six layers (typically named by Roman numerals). The input from 
the LGN is mainly sent to the layer IV. This layer is further subdivided into four 
sublayers (A, B, C$\alpha$ and C$\beta$). It is responsible for the initial processing
and is mainly specialized for monoocular input. The output is then sent mostly to the the 
layers II and III. These layers are functionally very similar and are typically labeled
as one layer II/III. Since this part the binocular processing prevails. The output is
then sent to the extrastriate areas like V2, V3 etc. that are responsible for the higher 
level visual processing. The rest of the layers I, V and VI are responsible mainly for 
the feedback and modulation of the signal. 

It is worth mentioning that mainly in layer IV there is a high range of interlayer connections.
Alongside with this in each layers there are also the inhibitory neurons that are 
also connected only in intra-layer manner and only affect the neurons in the same layer.
On the contrary some excitatory neurons does spread across the layers.

In the V1 there is still an important phenomenon present called the \emph{retinoscopy}.
It is the phenomenon where the neurons in the V1 are organized in the way that 
corresponds to the retinal organization. This means that the neurons closer to each
other in V1 correspond to neurons that are also close to each other in LGN and retina.
It is worth mentioning that the mapping is not uniformly precise since some parts of 
the visual field are densely covered than others (center and periphery of the visual field).

\subsection{Receptive field properties of V1 cells}
\label{subsec:receptive_field}
The neurons in V1 are specialized in several types responsible for the different properties
of the visual stimuli. These types creates overlapping interconnected maps that cover 
all of these properties. The properties are the following:

\begin{description}
    \item[Orientation selectivity:] Since layers IVC ($\alpha$ and $\beta$) the  
    orientation field stops to be circular and becomes elongated. The neurons in the 
    layer IVC are thus selective for the orientation of the visual stimuli. In other
    words they prefer the signal of the line in specific angle. In fact this preferred
    periodically changes across the V1 which results in similar orientation preferences
    in the close vicinity of the neurons. In terms of orientation selectivity 
    we also distinguish between \emph{simple} and \emph{complex} cells.
    \begin{description}
        \item[Simple cells:] These cells are selective for the orientation of the line 
        and the position of the line. In other words they are similar to ON and OFF cells
        from the previous layers. Its input is basically the sum several ON and OFF cells 
        from the LGN. 
        \item[Complex cells:] These cells are selective for the orientation of the line
        but anywhere in the orientation field. In other words they do not have any 
        ON and OFF regions. Its input is the sum of the simple cells.
    \end{description} 

    \item[Direction selectivity:] Some of the V1 neurons are specialized for the detection
    of motion in perpendicular direction.
    \item[Binocularity:] Till the layer IV the neurons are mainly monocular. This
    changes in the future layers. The neurons then combine the information from both eyes 
    and need to properly overlap the same information.
    \item[Other types:] The rest of the neurons are specialized for instance for motion, 
    depth, feedback or color.
\end{description}

\section{Extrastriate Visual Cortex}
\label{sec:extrastriate}
After the signal leaves the V1 it is further processed in the higher visual areas.
These areas are responsible for more complex visual processing and deficits in these 
typically result in some perceptual deficits but not the overall blindness.

