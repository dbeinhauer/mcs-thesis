

%%% Choose a language %%%

\newif\ifEN
\ENtrue   % uncomment this for english
%\ENfalse   % uncomment this for czech

%%% Configuration of the title page %%%

% \def\ThesisTitleStyle{mff} % MFF style
%\def\ThesisTitleStyle{cuni} % uncomment for old-style with cuni.cz logo
\def\ThesisTitleStyle{natur} % uncomment for nature faculty logo

% \def\UKFaculty{Faculty of Mathematics and Physics}
\def\UKFaculty{Faculty of Science}

\def\UKName{Charles University in Prague} % this is not used in the "mff" style

% Thesis type names, as used in several places in the title
% \def\ThesisTypeTitle{\ifEN BACHELOR THESIS \else BAKALÁŘSKÁ PRÁCE \fi}
\def\ThesisTypeTitle{\ifEN MASTER THESIS \else DIPLOMOVÁ PRÁCE \fi}
%\def\ThesisTypeTitle{\ifEN RIGOROUS THESIS \else RIGORÓZNÍ PRÁCE \fi}
%\def\ThesisTypeTitle{\ifEN DOCTORAL THESIS \else DISERTAČNÍ PRÁCE \fi}
% \def\ThesisGenitive{\ifEN bachelor \else bakalářské \fi}
\def\ThesisGenitive{\ifEN master \else diplomové \fi}
%\def\ThesisGenitive{\ifEN rigorous \else rigorózní \fi}
%\def\ThesisGenitive{\ifEN doctoral \else disertační \fi}
% \def\ThesisAccusative{\ifEN bachelor \else bakalářskou \fi}
\def\ThesisAccusative{\ifEN master \else diplomovou \fi}
%\def\ThesisAccusative{\ifEN rigorous \else rigorózní \fi}
%\def\ThesisAccusative{\ifEN doctoral \else disertační \fi}



%%% Fill in your details %%%

% (Note: \xxx is a "ToDo label" which makes the unfilled visible. Remove it.)
\def\ThesisTitle{Modeling spatio-temporal dynamics in primary visual cortex using deep neural network model}
\def\ThesisAuthor{Bc. David Beinhauer}
\def\YearSubmitted{2025}

% department assigned to the thesis
\def\Department{Department of Cell Biology}
% Is it a department (katedra), or an institute (ústav)?
\def\DeptType{Department}

\def\Supervisor{Mgr. Ján Antolík, Ph.D.}
\def\SupervisorsDepartment{Department of Cell Biology}

% Study programme and specialization
\def\StudyProgramme{Bioinformatics}
\def\StudyBranch{N-BINF}

\def\Dedication{%
I would like to express my sincere gratitude to my supervisor, 
Mgr. Ján Antolík, Ph.D., for his unwavering guidance and support throughout 
this project. Additionally, I am deeply thankful to Mgr. Luca Baroni for his valuable 
insights and feedback. Finally, I would like to extend my heartfelt gratitude
to my friends and family for their continuous support and encouragement throughout
my studies.
}

\def\AbstractEN{
    In this project, we develop a biologically inspired recurrent deep neural
    network (DNN) model of the primary visual cortex (V1) to study the 
    spatio-temporal dynamics of neuronal responses. The model incorporates key 
    biological constraints, including cortical layering, excitatory-inhibitory
    differentiation, and synaptic adaptation.
    Additionally, we introduce shared DNN modules trained to capture
    the complex non-linear activation functions of real-life neurons. Our
    dataset conists of artificially generated data from a spiking model of 
    the cat's V1. We train our model to replicate neuronal responses in
    layer 4 (L4) and layer L2/3 (L2/3) of V1 based on inputs from the 
    lateral geniculate nucleus (LGN). 
    Our study demonstrates strong agreement between our model and the spiking 
    model in terms of correlation and spatio-temporal dynamics. Furthermore, 
    incorporating more complex activation functions via the DNN module improves 
    model performance.
% ABSTRACT IS NOT A COPY OF YOUR THESIS ASSIGNMENT!
}

\def\AbstractCS{%
    Tato práce je zaměřena na vývoj biologicky inspirovaného modelu primární
    zrakové kůry (V1) s využitím hlubokých rekurentních neuronových
    sítí (DNN) pro analýzu časoprostorové dynamiky neuronálních odpovědí. 
    Model zahrnuje klíčová biologická omezení, jako jsou dělení do vrstev,
    diferenciace na excitační a inhibiční neurony a synaptická adaptace.
    Dále představujeme sdílené DNN moduly trénované pro zachycení komplexních
    nelineárních aktivačních funkcí skutečných neuronů. Naše data pocházejí z
    uměle generovaných dat z tzv. "spiking" modelu kočičí V1. Při trénování 
    našeho modelu cílíme na replikaci neuronálních odpovědí ve vrstvě 4 (L4) 
    a vrstvě L2/3 (L2/3) V1 na základě vstupů z laterálního genikulárního jádra
    (LGN). Výsledky naší studie ukazují na silnou shodu mezi oběma modely 
    vzhledem k korelaci neuronových odpovědí a časoprostorové dynamice. 
    Dále ukazujeme, že použití složitějších aktivačních funkcí neuronů s pomocí
    sdílených DNN modulů zlepšuje výsledky modelu.
}

% 3 to 5 keywords (recommended), each enclosed in curly braces.
% Keywords are useful for indexing and searching for the theses by topic.
\def\Keywords{%
{visual cortex}, {recurrent networks}, {model deep neural networks}
}

% If your abstracts are long and do not fit in the infopage, you can make the
% fonts a bit smaller by this setting. (Also, you should try to compress your abstract more.)
% Alternatively, consider increasing the size of the page by uncommenting the
% geometry modification in thesis.tex.
\def\InfoPageFont{}
%\def\InfoPageFont{\small}  %uncomment to decrease font size

\ifEN\relax\else
% If you are writing a czech thesis, you additionally need to fill in the
% english translation of the metadata here!
\def\ThesisTitleEN{\xxx{Thesis title in English}}
\def\DepartmentEN{\xxx{Name of the department in English}}
\def\DeptTypeEN{\xxx{Department}}
\def\SupervisorsDepartmentEN{\xxx{Superdepartment}}
\def\StudyProgrammeEN{\xxx{study programme}}
\def\StudyBranchEN{\xxx{study branch}}
\def\KeywordsEN{%
\xxx{{key} {words}}
}
\fi
