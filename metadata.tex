

%%% Choose a language %%%

\newif\ifEN
\ENtrue   % uncomment this for english
%\ENfalse   % uncomment this for czech

%%% Configuration of the title page %%%

% \def\ThesisTitleStyle{mff} % MFF style
\def\ThesisTitleStyle{cuni} % uncomment for old-style with cuni.cz logo
% \def\ThesisTitleStyle{natur} % uncomment for nature faculty logo
% 
% \def\UKFaculty{Faculty of Mathematics and Physics}
\def\UKFaculty{Faculty of Science}

\def\UKName{Charles University} % this is not used in the "mff" style

% Thesis type names, as used in several places in the title
% \def\ThesisTypeTitle{\ifEN BACHELOR THESIS \else BAKALÁŘSKÁ PRÁCE \fi}
\def\ThesisTypeTitle{\ifEN MASTER THESIS \else DIPLOMOVÁ PRÁCE \fi}
%\def\ThesisTypeTitle{\ifEN RIGOROUS THESIS \else RIGORÓZNÍ PRÁCE \fi}
%\def\ThesisTypeTitle{\ifEN DOCTORAL THESIS \else DISERTAČNÍ PRÁCE \fi}
% \def\ThesisGenitive{\ifEN bachelor \else bakalářské \fi}
\def\ThesisGenitive{\ifEN master \else diplomové \fi}
%\def\ThesisGenitive{\ifEN rigorous \else rigorózní \fi}
%\def\ThesisGenitive{\ifEN doctoral \else disertační \fi}
% \def\ThesisAccusative{\ifEN bachelor \else bakalářskou \fi}
\def\ThesisAccusative{\ifEN master \else diplomovou \fi}
%\def\ThesisAccusative{\ifEN rigorous \else rigorózní \fi}
%\def\ThesisAccusative{\ifEN doctoral \else disertační \fi}



%%% Fill in your details %%%

% (Note: \xxx is a "ToDo label" which makes the unfilled visible. Remove it.)
\def\ThesisTitle{Modeling spatio-temporal dynamics in primary visual cortex using deep neural network model}
\def\ThesisAuthor{Bc. David Beinhauer}
\def\YearSubmitted{2025}

% department assigned to the thesis
\def\Department{Department of Cell Biology}
% Is it a department (katedra), or an institute (ústav)?
\def\DeptType{Department}

\def\Supervisor{Mgr. Ján Antolík, Ph.D.}
\def\SupervisorsDepartment{}%Department of Cell Biology}

% Study programme and specialization
\def\StudyProgramme{Bioinformatics}
\def\StudyBranch{Bioinformatics}

\def\Dedication{%
I would like to express my sincere gratitude to my supervisor, Mgr. Ján Antolík, Ph.D., for his unwavering guidance and support throughout this project. I am also deeply grateful to Mgr. Luca Baroni for his valuable insights and constructive feedback. Finally, I extend my heartfelt thanks to all my friends, family members, and colleagues for their continuous support and encouragement throughout my studies.

Computational resources were provided by the e-INFRA CZ project (ID:90254), supported by the Ministry of Education, Youth and Sports of the Czech Republic.
}

\def\AbstractEN{
    Recent advances in computational neuroscience and machine learning have enabled increasingly sophisticated models of neuronal systems. However, standard deep neural networks (DNNs) often prioritize task performance over biological plausibility, limiting their ability to capture complex neural dynamics.

    In this thesis, we propose a novel approach that integrates biologically inspired constraints into recurrent neural network (RNN) architectures to model the primary visual cortex (V1). Using synthetic data generated from a biologically detailed spiking neural network (SNN) model of cat V1, we develop RNN architectures incorporating anatomical structure alignment, excitatory-inhibitory neuron differentiation, biologically motivated neuronal modules, and synaptic depression mechanisms.

    Our results show that RNN architectures without complex internal modules can predict mean neuronal responses but struggle to capture full system dynamics. Introducing shared DNN neuron modules improves dynamics slightly, while RNN-based neuron modules substantially enhance the temporal fidelity of predictions. Conversely, synaptic depression modules did not improve performance, likely due to computational constraints and suboptimal hyperparameter tuning.

    This work demonstrates the promise of combining deep learning with biological realism to bridge the gap between predictive accuracy and interpretability, laying the groundwork for future applications to real neural recordings.
}

\def\AbstractCS{
    Nedávné pokroky v oboru výpočetních neurověd a strojového učení umožnily vývoj komplexních modelů neurálních systémů. Nicméně, hluboké neurální sítě (DNN) často prioritizují celkovou úspěšnost v řešení problému nad biologickou interpretací, což limituje schopnost modelu zachytit komplexní neurální dynamiku.

    V rámci této práce navrhujeme nový způsob modelování kombinující biologické znalosti systému pro návrh modelů rekurentních neurálních sítí (RNN) primární zrakové kůry (V1). Za použití syntetických dat vygenerovaných z biologicky detailního impulzivního modelu neuronové sítě (SNN) kočičího V1, jsme vyvinuli RNN s anatomicky odpovídající architekturou rozlišující excitační a inhibiční neurony, používající biologicky motivované moduly neuronů a zapojující mechanismy synaptické deprese.

    Naše výsledky ukazují, že RNN síť bez použití komplexních modulů neuronů může predikovat průměrné neurální odpovědi, ale zaostává v zachycení komplexní dynamiky systému. Zapojení DNN modulů neuronů mírně zlepšuje dynamiku, nicméně výrazné zlepšení v predikci dynamiky systému přináší až zapojení RNN modulů neuronů. Naproti tomu moduly synaptické deprese nezlepšují dynamiku sítě, pravděpodobně kvůli výpočetním omezením a neoptimální volbě hyperparametrů.

    Tato práce vykazuje potenciál kombinace hlubokých neurálních sítí s biologickými znalostmi umožňující kombinaci přesnosti predikcí s interpretovatelností a vytváří základy pro budoucí aplikace na reálných experimentálních datech.
}

% 3 to 5 keywords (recommended), each enclosed in curly braces.
% Keywords are useful for indexing and searching for the theses by topic.
\def\Keywords{%
{visual cortex}, {recurrent networks}, {model deep neural networks}
}

% If your abstracts are long and do not fit in the infopage, you can make the
% fonts a bit smaller by this setting. (Also, you should try to compress your abstract more.)
% Alternatively, consider increasing the size of the page by uncommenting the
% geometry modification in thesis.tex.
\def\InfoPageFont{}
%\def\InfoPageFont{\small}  %uncomment to decrease font size

\ifEN\relax\else
% If you are writing a czech thesis, you additionally need to fill in the
% english translation of the metadata here!
\def\ThesisTitleEN{\xxx{Thesis title in English}}
\def\DepartmentEN{\xxx{Name of the department in English}}
\def\DeptTypeEN{\xxx{Department}}
\def\SupervisorsDepartmentEN{\xxx{Superdepartment}}
\def\StudyProgrammeEN{\xxx{study programme}}
\def\StudyBranchEN{\xxx{study branch}}
\def\KeywordsEN{%
\xxx{{key} {words}}
}
\fi
